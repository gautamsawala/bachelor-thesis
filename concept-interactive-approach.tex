%%%
% Concept for an Interactive Approach
%%%

\chapter{Konzept für das interaktive Layout von graph-basierten Softwarediagrammen}

% Was kann der Nutzer verändern?
% Was kann er tun?
% Arten der Manipulation auflisten
% Was ist neu an dem Ansatz?


%Algorithmus-Beschreibung
%Layout-Patterns ??
%	implizite und explizite
%	detaillierte Beschreibung der expliziten Layout-Patterns mit Bezug auf Arbeiten
%Worin unterscheidet sich mein Ansatz zu dem von Sonja Maier?
%Für welche Diagramme geeignet? (Klassendiagramme, Objektdiagramme, Komponentendiagramme, Graphen, Bäume)
%Vergleich zu bestehenden Ansätzen im Kapitel 3

% kombiniert Vorteile der freien Positionierung der Elemente im Diagramm und der automatischen Layout-Algorithmen miteinander kombiniert

% immediate feedback vs. delayed feedback [Wybrow S.69]
