%%%
% Automatic Layout
%%%

\section{Automatisches Layout}
\label{sec:automatic-layout}

% Was bedeutet automatisches Layout?
% Algorithmen zum Layout von Diagrammen
% Algorithmus, das auf ein Diagramm angewendet wird und das die Layout-Eigenschaften berechnet
% mit einem Knopfdruck im Nachhinein
% die Interaktion mit dem Diagramm fehlt
% kein direkter Einfluss auf das Ergebnis des Layout-Prozesses
% nicht geeignet für Änderungen im Diagramm -> muss nochmal gestartet werden
% (deutet darauf hin, dass) -> nicht intuitiv
% textbasiert auf textuelle Sprache vs. graphisch auf graphische Sprache
 
\subsection{Textbasierte Ansätze}

% Graphviz Dot DSL
%% Code-Ausschnitt
%% Layout-Eigenschaften (Parameter...)
%% textuell
%% mathematische Algorithmen für Layout von Graphen

% yUML.me, planttext.com
% Text-Notation in Form einer DSL
% fehlende Interaktion des Nutzers mit dem Diagramm
% Veränderungen an einer anderen Stelle, keine unmittelbare Bearbeitung des Diagramms
% kein Einfluss auf das Layout

\subsection{Graphische Ansätze}

% Unterstützung des automatischen Layouts in OmniGraffle
% Interaktion mit dem Diagramm wird von dem Layout-Algorithmus nicht berücksichtigt => Zerstören des mentalen Modells
% Zusatzfunktion in OmniGraffle und CASE-Tools (Visual Paradigm anschauen)
%% Graphviz Layout Engine

% SugiBib - ein Framework, das auf dem Sugiyama Algorithmus basiert und die Semantik und Strukturregeln berücksichtigt

\subsection{Zusammenfassung der Eigenschaften}

% Zusammenfassung
%% Verletzung der Semantik- und Strukturregeln durch die mathematischen Algorithmen
%% Interaktion im Vordergrund ist gewünscht
%% nicht geeignet für den Prozess des Zeichnens
%% Unterstützung der Diagrammtypen und deren Strukturregeln

%% Neuordnung des Layouts -> Zerstören der sekundären Notation [Seybold]
