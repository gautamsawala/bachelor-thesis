%%%
% Interactive Semi-automatic Layout
%%%

\section{Interaktives halbautomatisches Layout}
\label{sec:interactive-user-controlled-layout}

Wie bereits in diesem Kapitel präsentiert wurde, unterstützen die meisten Editoren zur Erstellung von Diagrammen Hilfefunktionen für das manuelle Layout (Abschnitt \ref{sec:manual-layout}) und eventuell auch automatische Layout-Algorithmen (Abschnitt \ref{sec:automatic-layout}). Das manuelle Layout ist sehr intuitiv, zeichnet sich durch die direkte Interaktion des Nutzers mit dem Diagramm aus, vermisst aber eine automatisierte Layout-Berechnung und Berücksichtigung der ästhetischen Prinzipien. Insbesondere hat die Semantik der Diagramme keinen Einfluss auf das Layout. Dieses Problem wird durch das automatische Layout bekämpft, das aber in dem Bereich der Interaktivität versagt \cite{GladischSchumann14Semi-Automatic}. Die Vorteile der Ansätze aus beiden genannten Kategorien lassen sich kombinieren und bilden eine neue Kategorie der Ansätze für die halbautomatische Layout-Unterstützung in interaktiven Umgebungen, die in diesem Abschnitt näher behandelt wird.



% dynamische Algorithmen, Sequenzen von Modifizierungen
% inkrementelle Erstellung -> automatische Ansätze nicht möglich, eher Sketching [Arvo02]
% ausgelegt für Änderungen [Gladisch, ...]
% nutzer-gesteuert
% Nutzer kann einige Aspekte des Layouts beeinflussen, z.B. durch direkte Positionierung der Knoten bzw. Kanten oder durch eine Form von Feedback [Arvo02]

\subsection{Constraint-basierte Ansätze}

% Maier Chapter 2.3 !!
% Screenshot von Dunnart

\subsection{Pattern-basierter Ansatz}

% DiaGraph
% Link auf die Videos

\subsection{Anwendungsspezifische Ansätze}

\subsubsection{Smart Layout in MindNode}

% anwendungspezifisch

\subsubsection{EditLens}

% Multi-Touch Layout Techniques (Alignment Guides) TUD ?

\subsection{Eigenschaften und Vergleich}

% Wahrnehmungsorganisation hat eine größere Priorität als reine Berücksichtigung der syntaktischen Ästhetik [Shieber]
% Beides Kombinieren [Shieber]

% Nur visualisieren vs. auch Editieren [Gladisch]
% Nutzer entwickeln ein mentales Modell des Diagramms, das soll bei den inkrementellen Updates berücksichtigt werden [Gladisch]











