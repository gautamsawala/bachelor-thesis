%%%
% Interactive Semi-automatic Layout
%%%

\section{Interaktives halbautomatisches Layout}
\label{sec:interactive-user-controlled-layout}

Wie bereits in diesem Kapitel präsentiert wurde, unterstützen die meisten Editoren zur Erstellung von Diagrammen Hilfefunktionen für das manuelle Layout (Abschnitt \ref{sec:manual-layout}) und eventuell auch automatische Layout-Algorithmen (Abschnitt \ref{sec:automatic-layout}). Das manuelle Layout ist sehr intuitiv, zeichnet sich durch die direkte Interaktion des Nutzers mit dem Diagramm aus, vermisst aber eine automatisierte Layout-Berechnung und Berücksichtigung der ästhetischen Prinzipien. Dieses Problem wird durch das automatische Layout bekämpft, das aber in dem Bereich der Interaktivität versagt \cite{GladischSchumann14Semi-Automatic}. Die Vorteile der Ansätze aus beiden genannten Kategorien lassen sich kombinieren und bilden eine neue Kategorie der Ansätze für die halbautomatische Layout-Unterstützung.

Diese Ansätze sind für Nutzer-bedingte Änderungen ausgelegt und ermöglichen eine inkrementelle Erstellung der Diagramme mit Bezug auf das Layout. Somit sind sie insbesondere für interaktive Umgebungen geeignet \cite{Arvo02Techniques, GladischSchumann14Semi-Automatic, Wybrow08Using}.


% dynamische Algorithmen, Sequenzen von Modifizierungen
% Nutzer kann einige Aspekte des Layouts beeinflussen, z.B. durch direkte Positionierung der Knoten bzw. Kanten oder durch eine Form von Feedback [Arvo02]

\subsection{Struktur-basierte Nutzer-gesteuerte Ansätze}

Die erste Kategorie der Ansätze für das interaktive halbautomatische Layout bilden die Struktur-basierten Nutzer-gesteuerten Ansätze, die dem Nutzer ermöglichen, das Layout des Diagramms durch Erstellung und Verwaltung von Strukturregeln\footnote{Ein Beispiel für eine solche Strukturregel ist die im Abschnitt \ref{subsubsec:alignment-and-distribution} beschriebene gleichmäßige Verteilung der ausgewählten Objekte in Relation zueinander.} zu beeinflussen. Diese Strukturregeln erinnern an die Hilfefunktionen für das manuelle Layout (siehe Abschnitt \ref{subsec:help-functions-for-manual-layout}), sind aber dahingegen persistent \cite{Wybrow08Using}. Sie werden bei der Berechnung des Layouts durch einen dynamischen Layout-Algorithmus berücksichtigt und eingehalten.

Die Struktur-basierten Nutzer-gesteuerten Ansätze lassen sich in zwei Gruppen unterteilen: in Constraint-basierte und Pattern-basierte Ansätze. Beide Gruppen werden im Folgenden vorgestellt.

\subsubsection{Constraint-basierte Ansätze}

Die Strukturregeln können mit Hilfe von Constraints umgesetzt werden. Ein valides Layout wird dadurch berechnet, in dem alle Constraints im Diagramm mit einem Constraintlöser ausgewertet werden.



% deklarativ
% welche Regeln sollen gelten anstatt wie
% Constraintlöser, verschiedene Typen (beschrieben in [Mai])

% Cassowary for layout of UI in Cocoa/iOS

% Dunnart: Screenshot, http://dunnart.org
% Nutzer kann Constraints erstellen und auf Teile des Diagramms anwenden

% Probleme mit der Performance, da der Algorithmus nach jeder kleinen Änderung laufen muss

\subsubsection{Pattern-basierter Ansatz}

%\cite{Maier12A-Pattern-based}
%\cite{MaierMinas13A-Pattern-based}

% DiaGen, DiaMeta: http://www.unibw.de/inf2/DiaGen/

% http://www.unibw.de/inf2/Personen/Wissen_Mitarbeiter/sonja/research/layoutframework
% Videos: http://www.sonjamaier.de/dyndraw/screencasts/graphEditor.mov & http://www.sonjamaier.de/dyndraw/screencasts/ecoreEditor.mov

% Tool, Framework ist noch nicht veröffentlich

\subsection{Anwendungsspezifische Ansätze}

\subsubsection{Smart Layout in MindNode}

% anwendungspezifisch
% Mindmaps 
% Screenshot der Funktion
% Möglich, weil spezifisch für die visuelle Sprache für Mindmaps (Bäume)

\subsubsection{EditLens}

% Multi-Touch Layout Techniques (Alignment Guides) TUD ?

\subsection{Eigenschaften und Vergleich}

% Wahrnehmungsorganisation hat eine größere Priorität als reine Berücksichtigung der syntaktischen Ästhetik [Shieber]
% Beides Kombinieren [Shieber]

% Nur visualisieren vs. auch Editieren [Gladisch]
% Nutzer entwickeln ein mentales Modell des Diagramms, das soll bei den inkrementellen Updates berücksichtigt werden [Gladisch]

% Der Nutzer kann das Diagramm unmittelbar bearbeiten
% das mentale Modell bleibt beibehalten
% ästhetische Regeln werden nicht eingehalten







