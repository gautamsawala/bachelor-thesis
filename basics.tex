%%%
% Basics
%%%

\chapter{Grundlagen}

\section{Begriffsklärung}

% Was ist ein graph-basiertes Diagramm? => Graph?
% Was ist Layout eines Diagramms? (besser lesbar und und verständlich)
% Layout-Eigenschaften der Knoten und Kanten

% Visual Language Editors [Maier 2.1]

% mentales Modell eines Diagramms [Gladisch, Maier]

%graph-basierte Diagramme
%bilden Graphen eine Basis für viele Typen der Softwarediagrammen.

\section{Ästhetische Prinzipien}

% Patterns
% Layout Considerations [SKK+93]

% Hierarchie
% Alignment (Ausrichten)
% zentrierter Inhalt
% Überlappen der Knoten
% Schneiden von Kanten

%Auflistung der Layout-Prinzipien
%	The Elements of UML 2.0 Style
%	Arbeiten zum automatischen Layout von Klassendiagrammen
%	Figure “Layout Considerations” from [SKK+93]
%allgemeine Graphen vs. Klassendiagramme

% Typen von Hierarchien in Klassendiagrammen
