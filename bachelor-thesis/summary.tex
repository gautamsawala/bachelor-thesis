%%%%%%%%%%%
% Summary %
%%%%%%%%%%%

\chapter{Zusammenfassung}
\label{chapter:summary}

\section{Fazit}

% In dieser Arbeit wurde kein vollständiger Ansatz gezeigt, sondern eine Richtung für eine mögliche Implementierung...
% kein allgemeiner Ansatz, bildet eine Grundlage und präsentiert mögliche Bedienungsmechanismen für die Entwicklung eines allgemeinen Ansatzes
% Dies wird der weiteren Forschung überlassen.
% basiert auf Einschränkungen der Möglichkeiten, die der Nutzer ausführen kann
% dadurch ist es möglich für konkrete Diagrammtypen spezielle Algorithmen einzusetzen
% Contributions

\section{Ausblick}
\label{sec:outlook}


% Definition auf einem Metamodell, speziell für Layout-Patterns

\subsection{Implementierung der wichtigen Aktionen im Prototyp}
\label{subsec:implementation-of-importent-actions}

% Auswahl von Elementen (Problem mit Verschiebung von mehreren Elementen gleichzeitig)
% Undo
% Löschen von ausgewählten Knoten/Kanten
% Ändern der Knoten für eine Kante
% Speichern und Öffnen

\subsection{Verbesserung der Unterstützung von Semantik der Elemente}
\label{improvement-for-semantics-support}

% Metamodell im Prototyp
% Metamodell auch für Layout-Patterns wie in [Maier] -> Wiederverwendung für mehrere Diagramm-Typen

\subsection{Erweiterung zu einem allgemeinen Algorithmus für Klassendiagramme}
\label{subsec:algorithm-generalization}

% Einsatz eines Constraintlösers oder eines ausgereiften Algorithmus wie [Maier]
% Constraintlöser für Patterns, da bestehende Layout-Engines zu spezifisch -> soll allgemein sein und ein richtiger Constraintlöser sein
% Pattern-Abhängigkeiten müssen manuell verwaltet werden
% IDLPatternsSolver mit einem Constraintlöser
% Prototyp: Implementierung der impliziten Layout-Patterns (für den Constraintlöser)

% unterstützte Notation erweitern
% Unterstützung von sekundären Knoten für UML-Notizen

% Möglichkeit der Interaktivität der Patterns (Visualisierung - z.B. nur während Drag&Drop sichtbar, Erstellen durch den Nutzer wie ein [Maier] ist denkbar)
% Patterns für Kanten
% Erlauben dem Nutzer selbst Patterns zu erstellen (Kombinieren mit automatischen je nach Semantik)

% Unterstützung für Cluster/Zonen
% Historie der Layouts für weitere Berechnungen
% weitere Eigenschaften der Mausbewegung wie Richtung/Geschwindigkeit

% Unterstützung für Multi-Touch wäre denkbar
