%%%%%%%%%%%
% Summary %
%%%%%%%%%%%

\chapter{Zusammenfassung}
\label{chapter:summary}

\section{Fazit}

Diese Arbeit setzt sich mit der Problematik der \textbf{Layout-Berechnung in Diagrammen} auseinander und betrachtet diese aus der Sicht der Interaktion und der Art der Layout-Unterstützung. Zu Beginn wurden bestehende Ansätze für das Layout von Diagrammen ausführlich untersucht und bzgl. der genannten Aspekte kategorisiert. Die durchgeführte Analyse und die Aufstellung der Anforderung zur Eignung für \textbf{agile Modellierung} haben eine Grundlage für einen neuartigen Ansatz geschaffen, dessen Entwicklung den Schwerpunkt dieser Arbeit bildet.

Es handelt sich um einen Ansatz für Editoren zur Modellierung von graphbasierten Diagrammen, der die \textbf{interaktive Bearbeitung} des Diagramms mit der \textbf{automatischen Layout-Be\-rech\-nung} kombiniert. Dies konnte dadurch erreicht werden, indem die Möglichkeit der expliziten Einstellung von Layout-Eigenschaften durch den Nutzer eingeschränkt wurde, ohne auf die Interaktivität komplett verzichten zu müssen. Dafür war allerdings der Einsatz von speziellen Bedienungsmechanismen notwendig, die das besondere Merkmal des präsentierten Ansatzes bilden.

Das in dieser Arbeit entwickelte Konzept beschreibt die allgemeine Funktionsweise des Ansatzes und dessen Bestandteile, u.a. die Interaktionsschleife, die Bedienungsmechanismen und die eigentliche Layout-Berechnung. Allerdings ist dieser Ansatz nicht universell einsetzbar und erfordert eine Anpassung für konkrete Diagrammtypen. Im Rahmen dieser Arbeit wurden zwei Layout-Algorithmen für \textbf{vereinfachte Klassendiagramme} entworfen, an den der Ansatz getestet wurde. Die Vertiefung bzw. der Entwurf von neuen Layout-Algorithmen stellen Gegenstände für weitere Forschung dar und werden in Abschnitt \ref{sec:outlook} diskutiert.

Schließlich wurde das entworfene Konzept in Form eines Werkzeugs für die Erstellung von Klassendiagrammen prototypisch umgesetzt. Des Weiteren wurde der Prototyp in einer Nutzerstudie an realen Nutzern evaluiert. Diese hat im Großen und Ganzen positive Ergebnisse geliefert. Insbesondere wurde gezeigt, dass die halbautomatische Layout-Berechnung einen Mehrwert schafft und dass die eingesetzten Mechanismen der Interaktion verständlich sind. Dadurch stellt sich heraus, dass der Ansatz ein Potenzial besitzt und prinzipiell in Werkzeugen für agile Modellierung eingesetzt werden könnte. Dafür wäre allerdings eine eventuelle Weiterentwicklung und eine Anpassung für konkrete Anwendungsfälle notwendig.

\section{Ausblick}
\label{sec:outlook}

In dieser Arbeit wurde ein initiales Konzept vorgestellt, welches in vielen Richtungen verbessert und erweitert werden kann. Insbesondere durch die Evaluation wurden Stellen entdeckt, die in der weiteren Entwicklung aufgearbeitet werden können. Sie werden in diesem Abschnitt im Detail dokumentiert.

\subsection{Implementierung von erweiterten Funktionen im Prototyp}

In dem Prototyp wurden ausschließlich die Funktionen implementiert, die das Validieren des Konzeptes ermöglicht haben. Das Ziel war es nicht, ein vollständiges Modellierungswerkzeug zu entwickeln. Natürlich bedeutet dies, dass der Prototyp viele Funktionen vermisst, die aus anderen Programmen bekannt sind. Wie bereits in Abschnitt \ref{subsec:user-study-evaluation} aufgeführt wurde, hat die Nutzerstudie gezeigt, welche Funktionen den Teilnehmern am meisten gefehlt haben. Im Einzelnen handelte es sich um weitere Möglichkeiten der Manipulation mit Elementen im Diagramm und um Funktionen zur Verbesserung der Benutzerfreundlichkeit. Eine Übersicht der Ergebnisse wurde in Abbildung \ref{fig:missed-prototype-functions} illustriert. Des Weiteren wäre die Möglichkeit des Speicherns von Dokumenten sehr nützlich. Obwohl die genannten Funktionen nicht direkt mit dem eigentlichen Konzept verbunden sind, wäre deren Implementierung in dem Prototyp wünschenswert.

\subsection{Verbesserung der Unterstützung der Syntax und Semantik}

Durch das Aufstellen des Kriteriums \ref{req:syntax-and-semantics} wurde gefordert, dass der entworfene Ansatz die Syntax und Semantik der Diagramme berücksichtigt. Diese Anforderung ist insbesondere wichtig, um die Anpassung der Layout-Berechnung für einzelne Diagrammtypen zu ermöglichen. Obwohl dieses Kriterium in den präsentierten konkreten Layout-Algorithmen eingehalten wurde, ist für die Weiterentwicklung der Algorithmen bzw. den Entwurf von neuen Algorithmen für weitere Diagrammtypen notwendig, eine Formalisierung der Syntax und Semantik einzuführen. Dafür müsste ein Metamodell für die Beschreibung der abstrakten und konkreten Syntax der visuellen Sprachen eingesetzt werden. Des Weiteren könnten die Layout-Patterns ähnlich wie in \cite{Maier12A-Pattern-based} anhand von Patterns-spezifischen Metamodellen definiert werden, um deren Wiederverwendung in Layout-Algorithmen für unterschiedliche Diagrammtypen zu gewährleisten.

\subsection{Erweiterung zu einem allgemeinen Algorithmus für Klassendiagramme}

% Einsatz eines Constraintlösers oder eines ausgereiften Algorithmus wie [Maier]
% Constraintlöser für Patterns, da bestehende Layout-Engines zu spezifisch -> soll allgemein sein und ein richtiger Constraintlöser sein
% Pattern-Abhängigkeiten müssen manuell verwaltet werden
% IDLPatternsSolver mit einem Constraintlöser
% Prototyp: Implementierung der impliziten Layout-Patterns (für den Constraintlöser)

% Neu-Berechnung nach jedem Layout-Übergang -> Performance?

% unterstützte Notation erweitern
% Unterstützung von sekundären Knoten für UML-Notizen

% Möglichkeit der Interaktivität der Patterns (Visualisierung - z.B. nur während Drag&Drop sichtbar, Erstellen durch den Nutzer wie ein [Maier] ist denkbar)
% Patterns für Kanten
% Erlauben dem Nutzer selbst Patterns zu erstellen (Kombinieren mit automatischen je nach Semantik)

% Unterstützung für Cluster/Zonen
% Historie der Layouts für weitere Berechnungen
% weitere Eigenschaften der Mausbewegung wie Richtung/Geschwindigkeit

%  ~~~ Unterstützung für Multi-Touch wäre denkbar
