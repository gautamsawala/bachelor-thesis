%%%%%%%%%%%
% Summary %
%%%%%%%%%%%

\chapter{Zusammenfassung}
\label{chapter:summary}

\section{Fazit}

Diese Arbeit setzt sich mit der Problematik der \textbf{Layout-Berechnung in Diagrammen} auseinander und betrachtet diese aus der Sicht der Interaktion und der Art der Layout-Unterstützung. Zu Beginn wurden bestehende Ansätze für das Layout von Diagrammen ausführlich untersucht und bzgl. der genannten Aspekte kategorisiert. Die durchgeführte Analyse und die Aufstellung der Anforderung zur Eignung für \textbf{agile Modellierung} haben eine Grundlage für einen neuartigen Ansatz geschaffen, dessen Entwicklung den Schwerpunkt dieser Arbeit bildet.

Es handelt sich um einen Ansatz für Editoren zur Modellierung von graphbasierten Diagrammen, der die \textbf{interaktive Bearbeitung} des Diagramms mit der \textbf{automatischen Layout-Be\-rech\-nung} kombiniert. Dies konnte dadurch erreicht werden, dass die Möglichkeit der expliziten Einstellung von Layout-Eigenschaften durch den Nutzer eingeschränkt wurde, ohne auf die Interaktivität komplett verzichten zu müssen. Dafür war allerdings der Einsatz von speziellen Bedienungsmechanismen notwendig, die das besondere Merkmal des präsentierten Ansatzes bilden.

Das in dieser Arbeit entwickelte Konzept beschreibt die allgemeine Funktionsweise des Ansatzes und dessen Bestandteile, u.a. die Interaktionsschleife, die Bedienungsmechanismen und die eigentliche Layout-Berechnung. Allerdings ist dieser Ansatz nicht universell einsetzbar und erfordert eine Anpassung für konkrete Diagrammtypen. Im Rahmen dieser Arbeit wurden zwei Layout-Algorithmen für \textbf{vereinfachte Klassendiagramme} entworfen, an den der Ansatz getestet wurde. Die Vertiefung bzw. der Entwurf von neuen Layout-Algorithmen stellen Gegenstände für weitere Forschung dar und werden in Abschnitt \ref{sec:outlook} diskutiert.

Schließlich wurde das entworfene Konzept in Form eines Werkzeugs für die Erstellung von Klassendiagrammen prototypisch umgesetzt. Des Weiteren wurde der Prototyp in einer Nutzerstudie an realen Nutzern evaluiert. Diese hat im Großen und Ganzen positive Ergebnisse geliefert. Insbesondere wurde gezeigt, dass die halbautomatische Layout-Berechnung einen Mehrwert schafft und dass die eingesetzten Mechanismen der Interaktion verständlich sind. Dadurch stellt sich heraus, dass der Ansatz ein Potenzial besitzt und prinzipiell in Werkzeugen für agile Modellierung eingesetzt werden könnte. Dafür wäre allerdings eine eventuelle Weiterentwicklung und eine Anpassung für konkrete Anwendungsfälle notwendig.

\section{Ausblick}
\label{sec:outlook}

In dieser Arbeit wurde ein initiales Konzept vorgestellt, welches in vielen Richtungen verbessert und erweitert werden kann. Insbesondere durch die Evaluation wurden Stellen entdeckt, die in der weiteren Entwicklung aufgearbeitet werden können. Sie werden in diesem Abschnitt im Detail dokumentiert.

\subsection{Implementierung von erweiterten Funktionen im Prototyp}
\label{subsec:extension-of-the-prototype}

In dem Prototyp wurden ausschließlich die Funktionen implementiert, die das Validieren des Konzeptes ermöglicht haben. Das Ziel war es nicht, ein vollständiges Modellierungswerkzeug zu entwickeln. Natürlich bedeutet dies, dass der Prototyp viele Funktionen vermissen lässt, die aus anderen Programmen bekannt sind. Wie bereits in Abschnitt \ref{subsec:user-study-evaluation} aufgeführt wurde, hat die Nutzerstudie gezeigt, welche Funktionen den Teilnehmern am meisten gefehlt haben. Im Einzelnen handelte es sich um weitere Möglichkeiten der Manipulation von Elementen im Diagramm und um Funktionen zur Verbesserung der Benutzerfreundlichkeit. Eine Übersicht der Ergebnisse wurde in Abbildung \ref{fig:missed-prototype-functions} illustriert. Des Weiteren wäre die Möglichkeit des Speicherns von Dokumenten sehr nützlich. Obwohl die genannten Funktionen nicht direkt mit dem eigentlichen Konzept verbunden sind, wäre deren Implementierung in dem Prototyp wünschenswert.

\subsection{Verbesserung der Unterstützung der Syntax und Semantik}
\label{subsec:syntax-and-semantics-support-improvements}

Durch das Aufstellen des Kriteriums \ref{req:syntax-and-semantics} wurde gefordert, dass der entworfene Ansatz die Syntax und Semantik der Diagramme berücksichtigt. Diese Anforderung ist insbesondere wichtig, um die Anpassung der Layout-Berechnung für einzelne Diagrammtypen zu ermöglichen. Obwohl dieses Kriterium in den präsentierten konkreten Layout-Algorithmen eingehalten wurde, ist für die Weiterentwicklung der Algorithmen bzw. den Entwurf von neuen Algorithmen für weitere Diagrammtypen notwendig, eine Formalisierung der Syntax und Semantik einzuführen. Dafür müsste ein Metamodell für die Beschreibung der abstrakten und konkreten Syntax der visuellen Sprachen eingesetzt werden. Des Weiteren könnten die Layout-Patterns ähnlich wie in \cite{Maier12A-Pattern-based} anhand von Patterns-spezifischen Metamodellen definiert werden, um deren Wiederverwendung in Layout-Algorithmen für unterschiedliche Diagrammtypen zu gewährleisten.

\subsection{Verallgemeinerung des Layout-Algorithmus für Klassendiagramme}
\label{subsec:generalization-for-class-diagrams}

Der in Abschnitt \ref{subsubsec:tree-layout-algorithm} vorgestellte baumbasierte Layout-Algorithmus ist für vereinfachte Klassendiagramme ausgelegt und unterstützt ausschließlich die Modellierung von Vererbungshierarchien. Die Beschränkung der Struktur auf Bäume konnte für die Implementierung des Algorithmus ausgenutzt werden. Damit ist eine Erweiterung für vollständige Klassendiagramme mit weiteren Maßnahmen verbunden. Insbesondere müsste die Zusammensetzung der Layout-Patterns überarbeitet werden, möglicherweise durch den Einsatz eines \textbf{Constraintlösers}. Dafür wäre es notwendig, sowohl die expliziten als auch die impliziten Layout-Patterns in Form von Constraints ausdrücken zu können. Wie zu erwarten müsste in diesem Fall auf den Aspekt der Performance geachtet werden.

Da es sich bei den vollständigen Klassendiagrammen aufgrund der Erweiterung der unterstützten Notation um komplexere graphbasierte Strukturen handelt, wäre eventuell eine Anpassung der \textbf{Behandlung von Layout-Patterns} notwendig. Es wäre denkbar, die Layout-Patterns während der Bearbeitung zu visualisieren\footnote{Um das Diagramm übersichtlich zu halten, könnten die Layout-Patterns nur während der Verschiebungsaktion eingeblendet werden.}. Des Weiteren könnte eine direkte Interaktion mit den Layout-Patterns unterstützt werden, z.B. durch die Verschiebung einer Klasse auf die visuelle Repräsentation des Layout-Patterns für das Hinzufügen der Klasse zu dem Layout-Pattern. Obwohl der in dieser Arbeit vorgestellte Ansatz eine vollautomatische Verwaltung von Layout-Patterns durch die Layout-Algorithmen vorsieht, könnte aufgrund der komplexeren Struktur der Diagramme sogar die Möglichkeit der Erstellung von Layout-Patterns durch den Nutzer\footnote{Je nach Semantik könnte dies mit der automatischen Erstellung von Layout-Patterns kombiniert werden.} ähnlich wie in \cite{Maier12A-Pattern-based} benötigt werden. Weiterhin führt die Unterstützung von mehreren Relationstypen zu der Notwendigkeit der Einführung von Layout-Patterns für Kanten, wodurch deren Routen bestimmt würden.

Außerdem ist es denkbar, den Ansatz um weitere Eingabequellen für die Layout-Algorithmen zu erweitern. Dazu zählen Eigenschaften der Mausbewegung wie z.B. die Richtung oder Geschwindigkeit oder eine Verwaltung der Historie von angewendeten Layouts.
