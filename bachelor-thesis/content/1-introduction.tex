%%%%%%%%%%%%%%%%
% Introduction %
%%%%%%%%%%%%%%%%

\chapter{Einleitung}
\label{chapter:introduction}

\section{Motivation}

Der Einsatz von Agilität in der Softwareentwicklung, bekannt unter dem Begriff \textbf{agile Softwareentwicklung}, erfreut sich heutzutage einer großen Beliebtheit. In der Umfrage \enquote{State of Agile} der Firma VersionOne aus dem Jahr 2013 haben 88\% der befragten Personen aus nordamerikanischen und europäischen Softwareentwicklungsunternehmen bestätigt, dass in ihren Organisationen agile Softwareentwicklung eingesetzt wird. Die Umfrage zeigt weiterhin eine steigende Tendenz dieses Anteils in den letzten drei Jahren \cite{VersionOne148th-Annual}.

Agile Softwareentwicklung basiert auf den Grundsätzen aus \cite{BeckBeedle01Manifest}. Im Vergleich zu den traditionellen und schwergewichtigen Softwareentwicklungsprozessen (wie z.B. \textit{Unified Process} oder dem Wasserfallmodell) sind die agilen schlanker, flexibler und für die Bedürfnisse der heutigen Zeit angepasst. Des Weiteren stellt die agile Softwareentwicklung das Schreiben von Quellcode in den Fokus und unterstützt dies durch zahlreiche Praktiken wie z.B. testgetriebene Entwicklung, \enquote{Code-Refactoring} oder \enquote{Code-Reviews}. Insbesondere bei kleinen Entwicklerteams und Einzelentwicklern, die Anwendungen für moderne Plattformen entwickeln\footnote{mobile Apps, Webanwendungen, Endbenutzer-Desktopanwendungen}, wird während der Entwicklung in der Regel auf Modellierung und umfassende Dokumentation mithilfe von UML-Dia\-gram\-men verzichtet.

Der agile Ansatz lässt sich jedoch auch auf die Softwaremodellierung übertragen. Die Grundlage dafür bildet die in \cite{Ambler02Agile} von Scott W. Ambler präsentierte Methodik \textbf{\textit{Agile Modeling}}. Diese auf Praktiken basierte Methodik zeigt, dass effektive Softwaremodellierung auch im agilen Umfeld Vorteile bringt. Des Weiteren wird empfohlen, mit den einfachsten Tools leichtgewichtige Modelle zu erstellen, die einen bestimmten Zweck erfüllen, wie z.B. einen Sachverhalt zu veranschaulichen oder die Kommunikation zwischen mehreren Teammitgliedern zu verbessern. Weiterhin soll keine zu komplizierte Notation\footnote{Der Autor rät dazu, eine hilfreiche Teilmenge der UML-Notation aber auch weitere Diagrammtypen und deren Notationen wie z.B. Flowcharts zu verwenden.} verwendet werden. Wenn der Inhalt der Modelle bereits im Code umgesetzt ist und die Modelle dem Projekt keinen Mehrwert mehr geben, sollen sie verworfen werden \cite{Ambler02Agile}.

Zum einen können die Modelle in Form von Skizzen auf \textbf{Papier oder Whiteboards} gezeichnet werden. Dies ist eine schnelle, praktische und unmittelbare Lösung. Leider ist sie für nachträgliche Änderungen und Korrekturen nicht geeignet, erschwert die Lesbarkeit und verfügt über keine syntaktische und semantische Überprüfung.

Zum anderen bietet sich für die Erstellung der Modelle der Einsatz eines \textbf{Zeichentools}\footnote{Beispiel eines Zeichentools ist \textit{OmniGraffle}, das in Abschnitt \ref{subsubsec:omnigraffle} näher behandelt wird.} an. Die Zeichentools beheben die Probleme des statischen Charakters von Papier und bringen zusätzlich einen Vorteil durch die Digitalisierung mit sich. Jedoch unterstützen sie in der Regel keine Semantik und oft ist auch die Unterstützung der Syntax mangelhaft. So muss z.B. in einigen Zeichentools das Klassenelement für Klassendiagramme mit einem Rechteck und mehreren Textfeldern zusammengebaut werden.

Die \textbf{CASE-Tools}\footnote{Computer-aided software engineering}, die für Softwaremodellierung konzipiert sind und sowohl die Syntax als auch die Semantik der Modellierungssprache unterstützen, können ebenfalls für die Erstellung der Modelle verwendet werden. Obwohl die Nutzung von CASE-Tools in der Methodik \textit{Agile Modeling} nicht untersagt ist, sind diese Werkzeuge für die Erzeugung von einfachen Modellen in der Regel oft zu komplex und schwergewichtig.

Es stellt sich heraus, dass verfügbare Software-Tools nur eine unbefriedigende Unterstützung für die agile Modellierung bieten. Ein für die agile Modellierung geeignetes Tools soll neben einer einfachen Modellerstellung und den Prinzipien aus \cite{Ambler02Agile} auch die syntaktischen und semantischen Aspekte der Modellierungssprache unterstützen. Weiterhin ist es wichtig, dass der Inhalt des Diagramms im Fokus steht. Schließlich soll die Interaktion mit dem Diagramm gefördert werden, unter anderem durch eine intelligente Layout-Unterstützung, die den Gegenstand dieser Arbeit bildet.

\section{Zielstellung}

In dieser Arbeit soll ein Konzept für eine interaktive und diagrammspezifische Layout-Un\-ter\-stüt\-zung für visuelle Editoren aus dem Desktop-Bereich\footnote{Die Desktop-Editoren zeichnen sich dadurch aus, dass sie auf Fenstern basieren und die Bedienung mit der Maus und Tastatur erfolgt. Andere Eingabemöglichkeiten wie z.B. Multi-Touch werden in dieser Arbeit nicht behandelt.} erarbeitet werden. Der Schwerpunkt wird dabei auf graphbasierte Softwarediagramme gelegt, insbesondere wird auf Klassendiagramme eingegangen. Das Ziel ist es, einen möglichen Ansatz für die interaktive Layout-Unterstützung in leichtgewichtigen Softwaremodellierungswerkzeugen vorzustellen.

Im Rahmen der Arbeit sollen bestehende Ansätze für das Layout von Diagrammen untersucht und nach relevanten Kriterien kategorisiert werden. Die durchgeführte Analyse soll bei dem Entwurf des Konzeptes berücksichtigt werden. Die vorgestellten Layout-Mechanismen sollen in Form eines Prototyps implementiert und anschließend mithilfe einer Nutzerstudie evaluiert werden.

\section{Rahmenbedingungen}
\label{sec:thesis-conditions}

Obwohl es bereits Ansätze für die Bearbeitung von Diagrammen mit fortgeschrittenen Bedienungsparadigmen wie Multi-Touch gibt (z.B. in \cite{FrischHeydekorn10Diagram}), beschäftigt sich diese Arbeit ausschließlich mit Ansätzen für das Layout von Diagrammen im Bereich der Desktop-Anwendun\-gen. Der Grund dafür ist, dass für die Erstellung und Bearbeitung von Diagrammen der Einsatz von Desktop-Systemen überwiegt.

\section{Aufbau der Arbeit}

Die Arbeit ist in sieben Kapitel unterteilt. Nach diesem einführenden Kapitel wird zunächst in Kapitel \ref{chapter:basics} auf die grundlegenden Begriffe bezüglich des Layouts von Diagrammen und dessen ästhetischen Prinzipien eingegangen. In Kapitel \ref{chapter:existing-approaches} werden bestehende Ansätze für das Layout von Diagrammen anhand von kommerziellen Anwendungen, Software-Bibliotheken und Forschungsprojekten kategorisiert und im Detail beleuchtet. Dem neu entwickelten Konzept für das interaktive und diagrammspezifische Layout von graphbasierten Softwarediagrammen widmet sich Kapitel \ref{chapter:presented-approach}, in dem seine Funktionsweise erklärt und die einzelnen Bestandteile vorgestellt werden. In Kapitel \ref{chapter:prototype} werden die umgesetzten Funktionen und die Architektur der prototypischen Implementierung beschrieben. Den Gegenstand von Kapitel \ref{chapter:evaluation} bildet die Evaluation des umgesetzten Konzeptes, die sich auf eine durchgeführte Nutzerstudie stützt und deren Ergebnisse präsentiert sowie auswertet. Anschließend wird die Arbeit in Kapitel \ref{chapter:summary} zusammengefasst und es werden Vorschläge für die weitere Entwicklung aufgeführt.
