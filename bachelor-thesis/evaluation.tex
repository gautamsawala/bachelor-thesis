%%%%%%%%%%%%%%
% Evaluation %
%%%%%%%%%%%%%%

\chapter{Evaluation}
\label{chapter:evaluation}

\section{Erfüllung der Kriterien}

\begin{enumerate}[label={K.\arabic*}]

\item
\label{eval:gui}
\textbf{GUI}

\item
\label{eval:interactivity}
\textbf{Interaktivität}

\item
\label{eval:immediate-feedback}
\textbf{Unmittelbares Feedback}

\item
\label{eval:editing-support}
\textbf{Förderung des Prozesses der Diagramm-Erstellung}

\item
\label{eval:mental-map}
\textbf{Erhaltung des mentalen Modells}

\item
\label{eval:focus-on-the-content}
\textbf{Förderung der Konzentration auf den Inhalt}

\item
\label{eval:aesthetics-criteria}
\textbf{Berücksichtigung der ästhetischen Prinzipien}

% ästhetische Prinzipien in K.7

\item
\label{eval:syntax-and-semantics}
\textbf{Berücksichtigung der Syntax und Semantik}

\item
\label{eval:user-friendly}
\textbf{Benutzerfreundlichkeit}

\end{enumerate}

% kleine Diagramme sind schnell zu zeichnen, manuelle Bearbeitung skaliert nicht mit der Größe des Diagramms \cite{Eichelberger05Aesthetics}
% -> Problem, wenn viele Klassen im Diagramm

\section{Nutzerstudie}

Um das umgesetzte Konzept zu validieren, wurde im Rahmen der Bachelor-Arbeit eine Nutzerstudie durchgeführt, die den Gegenstand für die folgenden Abschnitte bildet. Zunächst werden in Abschnitt \ref{subsec:user-study-setting} der Aufbau und die Durchführung im Detail erläutert. In Abschnitt \ref{subsec:user-study-evaluation} folgt eine Auswertung der Ergebnisse.

\subsection{Aufbau und Durchführung}
\label{subsec:user-study-setting}

An der Nutzerstudie haben in individuellen Sitzungen 7 Testpersonen teilgenommen. In der Gruppe der Testpersonen wurden beide Geschlechter ungefähr gleichmäßig vertreten (4 männliche und 3 weibliche Teilnehmer). Es handelte sich um 4 Studenten im durchschnittlichen Alter von 23 Jahren und 3 wissenschaftliche Arbeiter im durchschnittlichen Alter von 32 Jahren. Des Weiteren hatten 6 von 7 Teilnehmern einen Bezug zu Informatik.

Jede Sitzung hat mit einer kurzen Beschreibung des Themas der Bachelor-Arbeit angefangen. Danach wurde der Ablauf der Sitzung vorgestellt und der Teilnehmer wurde um das Ausfüllen des ersten Teils des vorgefertigten Fragebogens (siehe Abschnitt \ref{sec:user-study-material-questionnaire}) gebeten, in dem allgemeine Informationen und Vorkenntnisse abgefragt wurden. Nach Bedarf wurde anschließend eine kurze Einführung über Klassendiagramme gegeben, um die Notation und die wichtigsten Begriffe wie Klasse, Vererbung, Oberklasse und Unterklasse zu erklären.

%Die eigentlichen Tests wurden an dem entwickelten und im Kapitel \ref{chapter:prototype} beschriebenen Prototyp durchgeführt. 



%wurde demonstriert



%In individuellen Sitzungen wurden nach einer kurzen Einführung vorgeschriebene Aufgaben in dem Prototyp erledigt und anschließend durch die Testpersonen bewertet.

% Probandengruppe beschreiben (Alter, Studenten, Erfahrungen...)

% Hardware
% - Mac
% - Trackpad, Apple Magic Mouse oder klassische optische Maus zur Auswahl
% Software
% - Mac OS X
% - Prototyp
% - Bildschirmaufnahme mit Screeny (http://screenyapp.com)

% Ablauf der Session (siehe Mindmap)

% Einführung Prototyp, da Anlernen notwendig
% Beschreibung der möglichen Aktionen
% Aufgaben (am baumbasierten Layout)
% Verweis auf Material im Anhang

\subsection{Auswertung}
\label{subsec:user-study-evaluation}

% Beobachtungen
% Ergebnisse und Auswertung des Fragebogens
% Interpretation der Beobachtungen und Ergebnisse

% Fehlende Aktionen im Prototyp

\section{Zusammenfassung}
