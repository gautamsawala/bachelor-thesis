%%%%%%%%%%
% Basics %
%%%%%%%%%%

\chapter{Grundlagen}

\section{Begriffsklärung}

\subsection{Graph-basierte Diagramme}
\label{subsec:graph-based-diagrams}

% Was ist ein graph-basiertes Diagramm? => Graph?
% Graph-basierte Diagramme
% Node-Link Diagramme [Eic S.61]
% bilden Graphen eine Basis für viele Typen der Softwarediagrammen.
% Welche Diagramme z.B.? (Klassendiagramme, Objektdiagramme, Komponentendiagramme, Graphen, Bäume)

\subsection{Layout eines Diagramms}

% Was ist Layout eines Diagramms? (besser lesbar und verständlich)
% Layout-Eigenschaften der Knoten und Kanten

% Visual Language Editors [Maier 2.1]

\subsection{Mentales Modell}
\label{subsec:mental-map}

% mentales Modell eines Diagramms [Gladisch, Maier, Wybrow?]
\cite{Branke01Dynamic} 

\subsection{Layout-Patterns}

% 

\section{Ästhetische Prinzipien}
\label{sec:aesthetics-criteria}

% Patterns
% Layout Pattern [Maier]
% Layout Considerations [SKK+93]

% Hierarchie
% Alignment (Ausrichten)
% zentrierter Inhalt
% Überlappen der Knoten
% Schneiden von Kanten

%Auflistung der Layout-Prinzipien
%	The Elements of UML 2.0 Style
%	Arbeiten zum automatischen Layout von Klassendiagrammen
%	Figure “Layout Considerations” from [SKK+93]
%allgemeine Graphen vs. Klassendiagramme

% Typen von Hierarchien in Klassendiagrammen

% Ästhetische Prinzipien auflisten und zusammenfassen (Eichelberger)
