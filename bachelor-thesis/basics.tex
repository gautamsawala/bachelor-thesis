%%%%%%%%%%
% Basics %
%%%%%%%%%%

\chapter{Grundlagen}

Dieses Kapitel beschäftigt sich mit den Grundlagen für die Problematik der Layout-Berechnung in Diagrammen. Zunächst werden in Abschnitt \ref{sec:disambiguation} die wichtigsten Begriffe eingeführt und erläutert. Anschließend werden in Abschnitt \ref{sec:aesthetics-criteria} die ästhetischen Prinzipien für das Layout von allgemeinen Graphen und Klassendiagrammen der Notationssprache UML\footnote{Unified Modeling Language (\url{http://www.uml.org})} zusammengefasst.

\section{Begriffsklärung}
\label{sec:disambiguation}

\subsection{Graphbasierte Diagramme}
\label{subsec:graph-based-diagrams}

Viele Diagramme, die sich mit einer visuellen Sprache beschreiben lassen, basieren auf der Struktur von Graphen, d.h. sie werden durch Knoten (engl. Node) und Kanten (engl. Edge bzw. Link) gebildet \cite{Eichelberger05Aesthetics}. Die grafische Repräsentation der graphbasierten Diagramme (manchmal in der Literatur auch Node-Link-Diagramme genannt) besteht aus der Zusammensetzung von Textelementen und geometrischen Objekten wie Rechtecke, Ellipsen und Linien \cite{Wybrow08Using}. Die Struktur von Graphen kann um die Möglichkeit der Verschachtelung von Knoten und deren Verbindungen durch Kanten erweitert werden \cite{Siebenhaller03Automatisches, Wybrow08Using}. Trotz des häufigen Einsatzes dieser Erweiterung in vielen Diagrammtypen wird sie für die Zwecke dieser Bachelor-Arbeit außer Acht gelassen.

Zu den graphbasierten Diagrammen gehört eine große Anzahl an Softwarediagrammen, wie z.B. Klassendiagramme\footnote{Obwohl der UML-Standard eine Verschachtelung der Klassen in Pakete zulässt, sind an dieser Stelle einfachere Varianten der Klassendiagramme gemeint, wie z.B. die konzeptuellen Klassendiagramme nach \cite{Ambler04UML-2-Class}.}, Objektdiagramme bzw. Use-Case-Diagramme der Notationssprache UML, Netzwerkdiagramme oder Flowcharts.

\subsection{Layout}
\label{subsec:layout}

Durch die Spezifikationen der konkreten Diagrammtypen wird in der Regel ausschließlich die Struktur und die grafische Repräsentation festgelegt. Dagegen wird die Bestimmung der konkreten Eigenschaften der geometrischen Objekte wie z.B. ihre Anordnung nicht spezifiziert. Diese kann sich nach Konventionen, den Präferenzen der Nutzer bzw. nach ästhetischen Prinzipien richten \cite{Maier12A-Pattern-based, Wybrow08Using}. Durch eine Anpassung dieser Eigenschaften kann zum einen die Lesbarkeit des Diagramms verbessert werden und zum anderen können bestimmte Eigenschaften des Diagramms hervorgehoben werden \cite{Maier12A-Pattern-based}. Weiterhin kann durch das Layout eine sekundäre Notation geschaffen werden, die dem Diagramm eine zusätzliche Bedeutung gibt \cite{SeyboldGlinz03An-Effective}. In den meisten Diagramm-Editoren ist die Erstellung des Layouts dem Nutzer überlassen, sie können aber auch durch einen automatischen Algorithmus berechnet werden \cite{Maier12A-Pattern-based}. Die einzelnen Ansätze für das Layout von Diagrammen werden im Kapitel \ref{chapter:existing-approaches} vorgestellt.

Das Layout eines Diagramms wird als eine Zuordnung von Layout-Eigenschaften zu den Objekten des Diagramms definiert. Zu den bedeutsamen Layout-Eigenschaften der graphbasierten Diagramme (siehe Abschnitt \ref{subsec:graph-based-diagrams}) gehören vor allem die Positionen und Größen der Knoten und Routen der Kanten.

\subsection{Mentales Modell}
\label{subsec:mental-map}

Während der Arbeit mit einem Diagramm wird durch den Nutzer ein mentales Modell (engl. mental map) entwickelt, was sich durch seine Wahrnehmung der Struktur und der Bedeutung auszeichnet \cite{Branke01Dynamic, GladischSchumann14Semi-Automatic}.

Wenn das Inhalt bzw. das Layout des Diagramms geändert werden, muss sich der Nutzer an diese Änderung anpassen. Insbesondere in dem Fall einer automatische Änderung ist es sehr wichtig, diese für den Nutzer möglichst nachvollziehbar durchzuführen, so dass das mentale Modell erhalten bleibt \cite{Branke01Dynamic, Maier12A-Pattern-based}.

\section{Ästhetische Prinzipien}
\label{sec:aesthetics-criteria}

% Patterns
% Layout Pattern [Maier]
% Layout Considerations [SKK+93]

% Hierarchie
% Alignment (Ausrichten)
% zentrierter Inhalt
% Überlappen der Knoten
% Schneiden von Kanten

%Auflistung der Layout-Prinzipien
%	The Elements of UML 2.0 Style
%	Arbeiten zum automatischen Layout von Klassendiagrammen
%	Figure “Layout Considerations” from [SKK+93]
%allgemeine Graphen vs. Klassendiagramme

% Typen von Hierarchien in Klassendiagrammen

% Ästhetische Prinzipien auflisten und zusammenfassen (Eichelberger)
