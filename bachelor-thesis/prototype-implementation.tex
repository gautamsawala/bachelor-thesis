%%%%%%%%%%%%%%%%%%%%%%%%%%%%
% Prototype Implementation %
%%%%%%%%%%%%%%%%%%%%%%%%%%%%

\chapter{Prototyp-Implementierung}
\label{chapter:prototype}

% Anforderungen an den Prototypen?

% Das Layout-Framework von Sonja Maier konnte nicht als Grundlage verwendet werden, da nicht veröffentlicht

% kurze Beschreibung des Prototypen
% unterstützte Operationen

% Animation der Layout-Übergängen
% - implizite Core Animation Animation
% - Überführen von mehreren Animationen mit POP

% IDLLayoutEngine besitzt intern Referenzen auf den Inhalt des Diagramms und muss daher mit dem Diagramm synchronisiert werden (manuell) mit s.g. Layout-Events
% Layout-Events auflisten und mit Bildern erklären

% Löschen eines (oder mehreren Elementen) im Prototypen nicht unterstützt, nur hinzufügen und "rausfahren"

% aus zeitlichen Gründen wurde die Semantik nicht tiefgründig ausgearbeitet und wurde der weiteren Forschung überlassen
% IDLEdge -> gerichtete Kanten mit max. einem Elternknoten (z.B. Vererbung in Klassendiagrammen) -> Diagramm ist also azyklisch

% Liste der unterstützten ästhetischen Kriterien

% Beschreibung der App, Komponenten, genutzte API's, unterstützte Layout-Methoden, Koordinaten (Umrechnung), Drag and Drop

% - IDLPattern, IDLAlignmentPattern, IDLTShapePattern -> Klassendiagramm
% Wiederverwendung, Komposition in IDLTShapePattern

% IDLLayoutEngine & Subclasses (bzw. unterstützte Engines)

% Koordinaten-Konversion mit Bild (NSView vs. IDLLayoutEngine)
% zentrierte Koordinaten -> Förderung der Umsetzung des impliziten Patterns zur Zentrierung

% wie werden die impliziten Patterns im Prototypen implementiert

% Vernachlässigungen
% ==================
% - kein Metamodell
% - keine Unterscheidung der abstrakten und konkreten Syntax

% Schwachstellen
% ==============
% manchmal Probleme mit Animation bei D&D
% Pattern-Abhängigkeiten müssen manuell verwaltet werden
% IDLPatternsSolver mit einem Constraint Solver

% Entwicklungspotential
% =====================
% Implementierung der Auswahl von Elementen
% Unterstützung von weiteren Operationen
% Anpassung für Klassendiagramme
