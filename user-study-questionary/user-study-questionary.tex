%%%
% User Study Questionary
%%%

\documentclass[12pt,fleqn]{scrreprt}

% encoding & language
\usepackage{selinput}
\SelectInputMappings{}
\usepackage[T1]{fontenc}
\usepackage[ngerman]{babel}
\usepackage[babel, german=quotes]{csquotes}

% questionary 
\usepackage{paperandpencil}
\usepackage[top=2.5cm,bottom=2.5cm,left=1.5cm,right=1.5cm]{geometry}

\begin{document}

\begin{center}
{\huge Nutzerstudie zur Bachelor-Arbeit \enquote{Entwicklung eines Konzeptes für das interaktive und diagrammspezifische Layout von graph-basierten Softwarediagrammen}}
\end{center}

\vspace{2cm}

\question{\textbf{Geben Sie bitte Ihren Vornamen an.} \\ Diese Angabe ist \underline{freiwillig}. Die Ergebnisse der Nutzerstudie werden anonym veröffentlicht.}

\openone

\question{\bf Geben Sie bitte Ihr Alter an.}

\hfill\xxbox\hspace{6cm}

\question{\bf Geben Sie bitte Ihr Geschlecht an.}

\begin{answersB}
\item Männlich
\item Weiblich
\end{answersB}

\question{\bf Geben Sie bitte Ihren derzeitigen Berufsstatus an.}

\begin{answersB}
\item Berufstätig, Beruf: \linetext{}
\item Student, Studiengang: \linetext{}
\item Sonstiges
\end{answersB}

\question{\bf Haben Sie Erfahrung mit dem Betriebssystem Mac OS X?}

\begin{answersB}
\item Ja, ich nutze es täglich
\item Ja, ich habe es bereits benutzt
\item Nein
\end{answersB}

\question{\bf Verfügen Sie über Kenntnisse der objekt-orientierten Programmierung? Verstehen Sie Begriffe wie Klasse, Vererbung, Unterklasse und Oberklasse?}

\begin{answersB}
\item Ja
\item Nein
\end{answersB}

\question{\bf Haben Sie Erfahrung mit Klassendiagrammen der Notationssprache UML? Kennen Sie die Notation für Klassen und Vererbung in UML?}

\begin{answersB}
\item Ja
\item Nein
\end{answersB}

\question{\bf Welche Tools verwenden Sie zum Erstellen von (Software-)Diagrammen und in welchem Maß?}

\vertikalblockfive{sehr oft}{oft}{manchmal}{selten}{gar nicht}{
    \blocktextfive{\textbf{Diagramm-Programme} \\ (Microsoft Visio, OmniGraffle, ConceptDraw)}
    \blocktextfive{\textbf{Präsentationsprogramme} \\ (Apple Keynote, Microsoft PowerPoint)}
    \blocktextfive{\textbf{Online-Tools} \\ (draw.io, yUML.me, Lucidchart)}
    \blocktextfive{\textbf{UML-Editoren} \\ (Visual Paradigm, Magic\-Draw, ArgoUML)}
    \blocktextfive{\textbf{Andere Programme}\vspace{1cm}}
    \blocktextfive{\textbf{Stift und Papier}}
    \blocktextfive{\textbf{Whiteboard}\vspace{1cm}}
    \blocktextfive{\textbf{Sonstiges}}
}

\question{\bf Falls Sie andere als in der vorhergehenden Frage aufgelisteten Tools zum Erstellen von (Software-)Diagrammen verwenden, welche?}

\opentwo

\newpage

\question{\bf Bitte bewerten Sie folgende Aussagen bzgl. der Aufgaben:}

\vertikalblockfive{trifft völlig zu}{trifft zu}{teils/teils}{trifft nicht zu}{trifft gar nicht zu}{
    \blocktextfive{Ich habe meiner Meinung nach alle Aufgaben erfolgreich erfüllt.}
    \blocktextfive{Die Erfüllung der Aufgaben war anstrengend.}
    \blocktextfive{Ich denke, dass ich die Aufgaben auch ohne Vorstellung des Tools problemlos lösen konnte.}
}

\question{\bf Bitte bewerten Sie folgende Aussagen bzgl. der Bedienung und Interaktion:}

\vertikalblockfive{trifft völlig zu}{trifft zu}{teils/teils}{trifft nicht zu}{trifft gar nicht zu}{
    \blocktextfive{Ich hatte das Programm ganze Zeit unter Kontrolle.}
    \blocktextfive{Die Bedienung war intuitiv.}
    \blocktextfive{Die Bedienung war fehlertolerant.}
    \blocktextfive{Der Aufwand an der notwendigen Layout-Arbeit wurde durch die Einschränkung der freien Positionierung reduziert.}
    \blocktextfive{Das \underline{Layout} der modellierten Diagramme fand ich ästhetisch ansprechend.}
    \blocktextfive{Für mich war jederzeit erkennbar, an welche Stelle ich die angeklickte Klasse verschieben kann.}
    \blocktextfive{Die Bedienung war irritierend.}
}

\question{\bf Welche Aktion haben Sie in dem Prototypen vermisst?}

\opentwo

\question{\bf Geben Sie das größte Problem an, das Sie während der Erfüllung der Aufgaben beobachten konnten:}

\opentwo

\question{\bf Haben Sie weitere Anmerkungen/Kritikpunkte/Verbesserungsvorschläge?}

\openfour

\end{document}
