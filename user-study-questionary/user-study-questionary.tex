%%%
% User Study Questionary
%%%

\documentclass[10pt,fleqn]{scrreprt}

% encoding & language
\usepackage{selinput}
\SelectInputMappings{}
\usepackage[T1]{fontenc}
\usepackage[ngerman]{babel}
\usepackage[babel, german=quotes]{csquotes}

% questionary 
\usepackage{papi}
\usepackage[top=2.5cm,bottom=2.5cm,left=1.5cm,right=1.5cm]{geometry}

% questions counter
\newcounter{QuestionsCounter}
\newcommand{\questionnumber}{\stepcounter{QuestionsCounter}\arabic{QuestionsCounter}}
\newcommand{\numberedquestion}[1]{\question{\questionnumber.}{#1}}

\begin{document}

\begin{center}
{\huge Nutzerstudie zur Bachelor-Arbeit \enquote{Entwicklung eines Konzeptes für das interaktive und diagrammspezifische Layout von graph-basierten Softwarediagrammen}}
\end{center}

\begin{questionnaire}

\numberedquestion{Geben Sie bitte Ihr Alter an.}

\hybridshortanswer{Ich bin \twoboxes Jahre alt.}

\separator

\numberedquestion{Geben Sie bitte Ihr Geschlecht an.}

\shortanswer{Männlich}
\shortanswer{Weiblich}

\separator

\numberedquestion{Geben Sie bitte Ihren derzeitigen Berufsstatus an.}

\shortanswer{Berufstätig}
\shortanswer{Student}
\shortanswer{Sonstiges}

\separator

\numberedquestion{Geben Sie ggf. Ihr Beruf bzw. Studiengang an.}

\longline

\separator

\numberedquestion{Haben Sie bereits Kenntnisse über Klassendiagramme?}

\shortanswer{Ja}
\shortanswer{Nein}

\end{questionnaire}

\end{document}
