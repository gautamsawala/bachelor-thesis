%%%
% Interactive Semi-automatic Layout
%%%

\section{Interaktives halbautomatisches Layout}
\label{sec:interactive-semi-automatic-layout}

Wie bereits in diesem Kapitel präsentiert wurde, unterstützen die meisten Editoren zur Erstellung von Diagrammen Hilfefunktionen für das manuelle Layout (Abschnitt \ref{sec:manual-layout}) und eventuell auch automatische Layout-Algorithmen (Abschnitt \ref{sec:automatic-layout}). Das manuelle Layout ist sehr intuitiv, zeichnet sich durch die direkte Interaktion des Nutzers mit dem Diagramm aus, vermisst aber eine automatisierte Layout-Berechnung und Berücksichtigung der ästhetischen Prinzipien. Dieses Problem wird durch das automatische Layout bekämpft, das aber in dem Bereich der Interaktivität versagt \cite{GladischSchumann14Semi-Automatic}. Die Vorteile der Ansätze aus beiden genannten Kategorien lassen sich kombinieren und bilden eine neue Kategorie der Ansätze für die halbautomatische Layout-Unterstützung.

Diese Ansätze sind für Nutzer-bedingte Änderungen ausgelegt und ermöglichen eine inkrementelle Erstellung der Diagramme mit Bezug auf das Layout. Somit sind sie insbesondere für interaktive Umgebungen geeignet \cite{Arvo02Techniques, GladischSchumann14Semi-Automatic, Wybrow08Using}.


% dynamische Algorithmen, Sequenzen von Modifizierungen
% Nutzer kann einige Aspekte des Layouts beeinflussen, z.B. durch direkte Positionierung der Knoten bzw. Kanten oder durch eine Form von Feedback [Arvo02]

\subsection{Struktur-basierte Nutzer-gesteuerte Ansätze}

Die erste Kategorie der Ansätze für das interaktive halbautomatische Layout bilden die Struktur-basierten Nutzer-gesteuerten Ansätze, die dem Nutzer ermöglichen, das Layout des Diagramms durch Erstellung und Verwaltung von Strukturregeln\footnote{Ein Beispiel für eine solche Strukturregel ist die im Abschnitt \ref{subsubsec:alignment-and-distribution} beschriebene gleichmäßige Verteilung der ausgewählten Objekte in Relation zueinander.} zu beeinflussen. Diese Strukturregeln erinnern an die Hilfefunktionen für das manuelle Layout (siehe Abschnitt \ref{subsec:help-functions-for-manual-layout}), sind aber dahingegen persistent \cite{Wybrow08Using}. Sie werden bei der Berechnung des Layouts durch einen dynamischen Layout-Algorithmus berücksichtigt und eingehalten.

Die Struktur-basierten Nutzer-gesteuerten Ansätze lassen sich in zwei Gruppen unterteilen: in Constraint-basierte und Pattern-basierte Ansätze. Beide Gruppen werden im Folgenden vorgestellt.

\subsubsection{Constraint-basierte Ansätze}
\label{subsubsec:constraint-based-approaches}

Die Strukturregeln können mit Hilfe von Constraints umgesetzt werden. Ein valides Layout wird dadurch berechnet, in dem alle Constraints im Diagramm mit einem Constraintlöser ausgewertet werden.



% deklarativ
% welche Regeln sollen gelten anstatt wie
% Constraintlöser, verschiedene Typen (beschrieben in [Mai])

% Cassowary for layout of UI in Cocoa/iOS

% Dunnart: Screenshot, http://dunnart.org
% Nutzer kann Constraints erstellen und auf Teile des Diagramms anwenden

% Probleme mit der Performance, da der Algorithmus nach jeder kleinen Änderung laufen muss

\subsubsection{Pattern-basierter Ansatz}

Der in \cite{Maier12A-Pattern-based} und \cite{MaierMinas10Combination} präsentierte Pattern-basierte Ansatz für das Layout von Diagrammen drückt die Strukturregeln in Form von Layout-Patterns aus. Der Ansatz kann für beliebige visuellen Sprachen eingesetzt werden, die mit Meta-Modellen beschrieben werden können. Dabei besteht das sprachenspezifische Meta-Modell aus zwei Teilen, nämlich der Meta-Modelle für die abstrakte und konkrete Syntax. Das zuletzt genannte Meta-Modell beschreibt die visuellen Eigenschaften der Sprache und wird daher durch die Berechnung des Layouts beeinflusst.

Die Beschreibung der Layout-Patterns erfolgt allerdings nicht mit Hilfe der sprachenspezifischen Meta-Modellen, sondern mit sprachenunabhängigen Pattern-spezifischen Meta-Model\-len\footnote{In \cite{Maier12A-Pattern-based} werden u.a. Pattern-spezifische Meta-Modelle für Mengen von Elementen, geordnete Liste von Elementen oder Graphen vorgestellt.}. Bei der Instanziierung eines Layout-Patterns wird das sprachenspezifische Modell auf ein Pattern-spezifisches Modell abgebildet\footnote{Dieses Sachverhalt wird mit Hilfe eines Beispiels in \cite[S.59ff]{Maier12A-Pattern-based} anschaulich gemacht.}. Dadurch wird eine mögliche Wiederverwendung der Layout-Patterns für diverse visuellen Sprachen gewährleistet. Weiterhin zeichnet sich der Pattern-basierte Ansatz dadurch aus, dass die Layout-Patterns verschiedene Ansätze zum Layout von Diagrammen u.a. Algorithmen zum Graphenzeichnen (siehe Abschnitt \ref{subsubsec:graph-drawing-tools}) und Constraints-basierte Algorithmen (siehe Abschnitt \ref{subsubsec:constraint-based-approaches}) kombinieren und sich zu Nutze machen.


% Auflistung der Layout-Patterns: tree layout, layered layout, equal distance, alignment…

%Kontroll-Algorithmus läuft im Hintergrund nach jeder Änderung des Diagramms bzw. der Instanziierung eines Layout-Patterns

%nutzer-gesteuert, nicht eingeschränkt in Interaktion

%Wie bereits erwähnt, ist dieser Ansatz für interaktive


% Videos: http://www.sonjamaier.de/dyndraw/screencasts/graphEditor.mov & http://www.sonjamaier.de/dyndraw/screencasts/ecoreEditor.mov

Dieser Ansatz wurde in Form eines Layout-Frameworks\footnote{\url{http://www.unibw.de/inf2/Personen/Wissen_Mitarbeiter/sonja/research/layoutframework}} implementiert, das bisher noch nicht veröffentlicht wurde. Dennoch wurde es in visuellen Editoren eingesetzt, die mit Hilfe von DiaMeta\footnote{Ein Framework zur Generierung von Editoren für visuelle Sprachen basierend auf Spezifikationen mittels Meta-Modelle. \url{http://www.unibw.de/inf2/DiaGen/}} bzw. Graphical Editing Framework\footnote{\url{http://www.eclipse.org/gef/}} erzeugt wurden \cite{Maier12A-Pattern-based}.

\subsection{Anwendungsspezifische Ansätze}

\subsubsection{Smart Layout in MindNode}

% anwendungspezifisch
% Mindmaps 
% Screenshot der Funktion
% Möglich, weil spezifisch für die visuelle Sprache für Mindmaps (Bäume)

\subsubsection{EditLens}

% Multi-Touch Layout Techniques (Alignment Guides) TUD ?

\subsection{Eigenschaften und Vergleich}

% Wahrnehmungsorganisation hat eine größere Priorität als reine Berücksichtigung der syntaktischen Ästhetik [Shieber]
% Beides Kombinieren [Shieber]

% Nur visualisieren vs. auch Editieren [Gladisch]
% Nutzer entwickeln ein mentales Modell des Diagramms, das soll bei den inkrementellen Updates berücksichtigt werden [Gladisch]

% Der Nutzer kann das Diagramm unmittelbar bearbeiten
% das mentale Modell bleibt beibehalten [preserving the mental map (in dynamic context) instead of aesthetics (in static context)]
% ästhetische Regeln werden nicht eingehalten







